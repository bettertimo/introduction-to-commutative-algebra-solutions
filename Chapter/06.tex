\documentclass[../A&M.tex]{subfiles}

\begin{document}

\chapter{Chain Conditions}

\subsection*{Exercise 6.1}

(i) Consider the ascending chain $\ker u \sbe \ker u^2 \sbe \cdots$ in $M$. As $M$ is Noetherian, we have $\ker u \sbe \cdots \sbe \ker u^n = \ker u^{n+1} = \cdots$ for some $n$. We claim that $\ker u^n = 0$. (Then $\ker u = 0$ and we are done.)

Let $x\in\ker u^n$. Choose $y_1$ s.t. $u(y_1) = x$. And choose $y_i$ inductively s.t. $u(y_{i+1}) = y_i$. In particular, $u(y_n) = y_{n-1} \implies u^n(y_n) = u^{n-1}(y_{n-1}) = \cdots = u(y_1) = x \implies u^{2n}(y_n) = u^n(x) = 0 \implies y_n \in \ker u^{2n} = \ker u^n \implies u^n(y_n) = x = 0$. This shows that $\ker u^n = 0$.

(ii) Consider the descending chain $\im u \spe \im u^2 \spe \cdots$ in $M$. As $M$ is Artinian, we have $\im u \spe \cdots \spe \im u^n = \im u^{n+1} = \cdots$ for some $n$. We claim that $\im u^n = M$. (Then $\im u = M$ and we are done.)

Let $x\in M$. Then $u^n(x) \in \im u^n = \im u^{2n} \implies u^n(x) = u^{2n}(y) = u^n(u^n(y))$ for some $y\in M$. As $u$ is injective, we have $x = u^n(y) \in \im u^n$.  This shows that $\im u^n = M$.

\subsection*{Exercise 6.2}

Let $N$ be a submodule of $M$. Define $\Sigma := \{ Ax_1+\cdots+Ax_n \mid x_i \in N \} \neq \emptyset$. By assumption there's a maximal element in $\Sigma$, say $N'$. Note that $N' \sbe N$. If $N'\subset N$, take $y \in N \setminus N'$. Then $N'+Ay \in \Sigma$ and $N'+Ay \supset N'$, a contradiction. So $N' = N$. In particular, $N$ is finitely generated. This shows that $M$ is Noetherian by (6.2).

\subsection*{Exercise 6.3}

Consider the exact sequence
$$
0 \lra N_1/(N_1\cap N_2) \lra M/(N_1\cap N_2) \lra M/N_1 \lra 0
$$
Note that by (6.3) $N_1/(N_1\cap N_2) \simeq (N_1+N_2)/N_2$ is Noetherian (Artinian) as $(N_1+N_2)/N_2$ is a submodule of the Noetherian (Artinian) module $M/N_2$. Hence, $M/(N_1\cap N_2)$ is Noetherian (Artinian) by (6.3). 

\subsection*{Exercise 6.4}

By (6.2) we know $M$ is a finitely generated $A$-module. Let $x_1,\ldots,x_n$ be a set of generators. Consider the $A$-module homomorphism $\phi: A \to M^n$ given by $\phi(a) := (ax_1,\ldots,ax_n)$. It's easy to check that $\ker \phi = \afk$. So $A/\afk$ is isomorphic to a submodule of the $A$-module $M^n$. As $M^n$ is a Noetherian $A$-module by (6.4), so is $A/\afk$. And this implies that $A/\afk$ is Noetherian as an $(A/\afk)$-module.

The property Noetherian can not be replaced by Artinian. Consider Example 3 in page 74. The abelian group $G$ is Artinian as a $\ZZ$-module with the annihilator $\afk=0$. But $\ZZ/\afk = \ZZ$ is not an Artinian ring by Example 2.

\subsection*{Exercise 6.5}

Let $Y\sbe X$ be a subspace. Given an ascending chain $\{Y_i\}$ of open sets in $Y$. Write $Y_i = X_i \cap Y$ for some open set $X_i$ in $X$. Consider the ascending chain $\{\bigcup_{i=1}^n X_i\}$ of open sets in $X$. Since $X$ is Noetherian, there exists $n\in\NN$ s.t. $\bigcup_{i=1}^n X_i = \bigcup_{i=1}^m X_i \spe X_m$ for all $m\geq n$. This implies that
$$
Y_n = \bigcup_{i=1}^n Y_i = \bigcup_{i=1}^n (X_i \cap Y) = \left( \bigcup_{i=1}^n X_i \right) \cap Y \spe X_m \cap Y = Y_m
$$
for all $m\geq n$. So $Y_n = Y_{n+1} = \cdots$ and hence $\{Y_i\}$ is stationary.

Let $\{U_i\}$ be an open covering of $X$. Set $\Sigma := \{ \bigcup_{i=1}^n U_i \mid n\in\NN \}$. By assumption $\Sigma$ has a maximal element, say $U := \bigcup_{i=1}^n U_i$. If $U \subset X$, take $x \in X \setminus U$. As $\{U_i\}$ covers $X$, $x\in U_j$ for some $j \notin \{1,\ldots,n\}$. Then $U \cup U_j \in \Sigma$ and $U \cup U_j \supset U$, a contradiction. This shows that $U = \bigcup_{i=1}^n U_i = X$. Hence, $X$ is quasi-compact.

\subsection*{Exercise 6.6}

(i) $\Rightarrow$ (iii) Immediately follows from Exercise 6.5.

(iii) $\Rightarrow$ (ii) Nothing to prove.

(ii) $\Rightarrow$ (i) Given an ascending chain $\{U_i\}$ of open sets in $X$. Consider the open subspace $U:=\bigcup_{i=1}^\infty U_i$ in $X$. Note that each $U_i$ is open in $U$, so $\{U_i\}$ is itself an open covering of $U$. As $U$ is quasi-compact, there exists $n\in\NN$ s.t. $U = \bigcup_{j=1}^n U_{i_j}$. Let $N:=\max\{i_1,\ldots,i_n\}$, then $U = \bigcup_{i=1}^\infty U_i = U_N$. So the chain $\{U_i\}$ is stationary from $U_N$ and hence $X$ is Noetherian.

\subsection*{Exercise 6.7}

Let $X$ be a Noetherian space and suppose $X$ is not a finite union of irreducible closed subspaces. Consider the set $\Sigma$ of non-empty closed subsets of $X$ which are not finite unions of irreducible closed subspaces. $\Sigma \neq \emptyset$ because $X\in\Sigma$. As $X$ is Noetherian, there exists a minimal element $Y\in\Sigma$. Note $Y$ is not irreducible. So there exist non-empty open sets $U_1,U_2$ in $Y$ s.t. $U_1 \cap U_2 = \emptyset$. Set $Y_i := Y \setminus U_i \subset Y$, which are non-empty and closed in $Y$. Then $Y = Y_1 \cup Y_2$. By the minimality of $Y$, we know $Y_1,Y_2$ are both finite unions of irreducible closed subspaces. Hence so is $Y$, a contradiction.

Since $X$ is a finite union of irreducible closed subspaces, so by Exercise 1.20(ii), $X = \bigcup_{i=1}^n X_i$ where each $X_i$ is maximal and irreducible (i.e., an irreducible component of $X$). Let $Y$ be any irreducible component of $X$. Then
$$
Y = X \cap Y = \left( \bigcup_{i=1}^n X_i \right) \cap Y = \bigcup_{i=1}^n (X_i \cap Y)
$$
Each $X_i \cap Y$ is closed in $Y$. And as $Y$ is irreducible, $Y = X_i \cap Y \sbe X_i$ for some $i$. This implies that $Y = X_i$ as they are both irreducible components. Hence $X_1,\ldots,X_n$ are all the irreducible components of $X$.

\subsection*{Exercise 6.8}

\subsection*{Exercise 6.9}

\subsection*{Exercise 6.10}

\subsection*{Exercise 6.11}

\subsection*{Exercise 6.12}
\phantom{}

\end{document}