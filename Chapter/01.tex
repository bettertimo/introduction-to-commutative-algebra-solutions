\documentclass[../A&M.tex]{subfiles}

\begin{document}

\chapter{Rings and Ideals}

\subsection*{Exercise 1.1}

Take an odd number $n$ s.t. $x^n=0$. Then
$$
(1+x)(x^{n-1}-x^{n-2}+\cdots-x+1)=x^n+1=1
$$
So $1+x$ is a unit.

Given a nilpotent element $x$ and a unit $u$ with the inverse $v$. Then $x+u=u(vx+1)$ is a unit. (Note that $vx$ is still nilpotent and so $vx+1$ is a unit by the first step.)

\subsection*{Exercise 1.2}

(i) $(\Rightarrow)$ Suppose $g(x)=b_0+b_1x+\cdots+b_mx^m$ is the inverse of $f(x)$. Then we have $f(x)g(x)=1$. Following the hint, we first show by induction on $r$ that $a_n^{r+1} b_{m-r}=0$ for all $r=0,1,\ldots,m$.

Clearly when $r=0$, we have $a_n b_m=0$ by looking at the coefficient of $x^{n+m}$ in the equation $f(x)g(x)=1$. Suppose now $a_n b_m = a_n^2 b_{m-1} = \cdots = a_n^r b_{m-(r-1)} = 0$. We consider $a_n^{r+1} b_{m-r}$. By looking at the coefficient of $x^{n+m-r}$ in the equation $a_n^r f(x)g(x) = a_n^r$ very carefully, we have
$$
a_n^{r+1} b_{m-r} + a_{n-1} a_n^r b_{m-r+1} + \cdots + a_{n-r} a_n^r b_m = 0
$$
This implies the first term $a_n^{r+1} b_{m-r}=0$ by the induction hypothesis. And this completes our claim. (A note on the above equation: There's no guarantee that $n-r\geq0$, but if the index of $a_i$ is negative then $a_i$ is naturally viewed as zero.)

To show our desired result, we proceed by induction on $n$. When $n=0$ there's nothing to prove. So suppose the statement holds for any unit in $A[x]$ with degree equal to $n-1$. Consider when $\deg f(x)=n$. By our claim, taking $r=m$ gives us $a_n^{m+1} b_0=0$. And since $a_0b_0=1$ we know $b_0$ is just a unit. So we have $a_n^{m+1}=0$ and hence $a_n$ is nilpotent.

As $a_n$ is nilpotent, so is $-a_nx^n$. Then by Exercise 1.1, $f(x)-a_nx^n = a_0+a_1x+\cdots+a_{n-1}x^{n-1}$ is a unit. And by the induction hypothesis, we have $a_0$ is a unit and $a_1,\ldots,a_{n-1}$ are also nilpotent.

$(\Leftarrow)$ Since $a_1,\ldots,a_n$ are nilpotent, so are $a_1x,\ldots,a_nx^n$. Hence $f(x)=a_0+a_1x+\cdots+a_nx^n$ is a unit by using Exercise 1.1 repeatedly.

(ii) $(\Rightarrow)$ Suppose $f(x)^k=0$ for some $k\in\NN$, then by considering the constant term of $f(x)^k$ we have $a_0^k=0$ and so $a_0$ is nilpotent. Moreover, since $f$ is nilpotent, so by Exercise 1.1, we have $1+f(x) = (1+a_0)+a_1x+\cdots+a_nx^n$ is a unit. And then by (i) we have $a_1,\ldots,a_n$ are nilpotent.

$(\Leftarrow)$ Since $a_0,a_1,\ldots,a_n$ are nilpotent, so are $a_0,a_1x,\ldots,a_nx^n$. Hence $f(x)=a_0+a_1x+\cdots+a_nx^n$ is nilpotent by (1.7).

(iii) $(\Rightarrow)$ Suppose $g(x)=b_0+b_1x+\cdots+b_mx^m$ is a polynomial of least degree $m$ s.t. $f(x)g(x)=0$. Following the hint, we first show by induction on $r$ that $a_{n-r}g(x)=0$ for all $r=0,1,\ldots,n$.

When $r=0$. Note that we have $a_n b_m=0$ by looking at the coefficient of $x^{n+m}$ in the equation $f(x)g(x)=0$. Then since $a_ng(x)f(x)=0$ and $\deg a_ng(x)<m$, so $a_ng(x)=0$ by the assumption on $g(x)$.

Suppose now $a_ng(x) = a_{n-1}g(x) = \cdots = a_{n-(r-1)}g(x) = 0$. (In particular, all their coefficients are zero.) We consider $a_{n-r}g(x)$. By looking at the coefficient of $x^{n+m-r}$ in the equation $f(x)g(x) = 0$ very carefully, we have
$$
a_n b_{m-r} + \cdots + a_{n-(r-1)} b_{m-1} + a_{n-r} b_m = 0
$$
This implies the last term $a_{n-r} b_m=0$ by the induction hypothesis. Then again, since $a_{n-r}g(x)f(x)=0$ and $\deg a_{n-r}g(x)<m$, so $a_{n-r}g(x)=0$ by the assumption on $g(x)$. This completes our claim. (A note on the above equation: Just like (i), there's no guarantee that $m-r\geq0$, but if the index of $b_j$ is negative then $b_j$ is naturally viewed as zero.)

From our claim, we have that $a_ib_j=0$ for all $i,j$. To show our desired result, we take a non-zero coefficient $b_j\in A$ of $g(x)$. Then $b_jf(x)=0$.

$(\Leftarrow)$ Nothing to prove.

(iv) Let $\afk_f,\afk_g,\afk_{fg}$ be the ideals generated by the coefficients of $f,g,fg$, respectively. Note that since $\afk_{fg}\sbe \afk_f,\afk_g$, so if $fg$ are primitive then clearly so are $f$ and $g$.

Conversely, assume $f$ and $g$ are primitive. Suppose on the contrary that $fg$ is not primitive. This means $\afk_{fg}\subset A$ so it's contained in a maximal ideal $\mfk$ in $A$. We consider everything modulo $\mfk$ and note that $(A/\mfk)[x]$ is a PID. Then we have $\ovl{fg} = 0 \in (A/\mfk)[x]$ but neither of $\ovl{f},\ovl{g}$ is zero in $(A/\mfk)[x]$ as $\mfk \subset \afk_f,\afk_g = A$. This is a contradiction.

\subsection*{Exercise 1.3}

\subsection*{Exercise 1.4}

Since the Jacobson radical $\Jcal(A[x])$ (resp. nilradical $\Ncal(A[x])$) of $A[x]$ is the intersection of all maximal (resp. prime) ideals in $A[x]$, and all maximal ideals are necessarily prime ideals, so clearly $\Ncal(A[x]) \sbe \Jcal(A[x])$.

Conversely, given $f(x) = a_0+a_1x+\cdots+a_nx^n \in \Jcal(A[x])$. Then by (1.9), $1-f(x) \cdot (-x) = 1+a_0x+a_1x^2+\cdots+a_nx^{n+1}$ is a unit in $A[x]$. So by Exercise 1.2 (i), $a_0,\ldots,a_n$ are nilpotent. Finally, by Exercise 1.2 (ii), $f(x)$ is nilpotent. So we have $f(x)\in \Ncal(A[x])$ and hence $\Jcal(A[x]) \sbe \Ncal(A[x])$.

\subsection*{Exercise 1.5}

(i) If $f$ is a unit then there exists $g=\sum b_mx^m$ s.t. $fg=1$. So we have $a_0b_0=1$. Conversely, if $a_0^{-1}$ exists in $A$, then one can use long division algorithm to compute $1/f$ in $A[[x]]$. So $f$ is a unit.

(ii) Suppose $f(x)=\sum a_nx^n$ is nilpotent but there exists some $a_n$ which is not nilpotent. Then take $N := \min\{ n\geq0 \mid a_n \text{ is not nilpotent}\}$. Since $a_0,\ldots,a_{N-1}$ are nilpotent, we have $f(x)-\sum_{i=0}^{N-1} a_nx^n = \sum_{i\geq N} a_nx^n$ is also nilpotent by (1.7). So there exists $k\in\NN$ s.t. $(\sum_{i\geq N} a_nx^n)^k=0$. But by considering the lowest degree term this implies that $a_N^k=0$, which is a contradiction.

The converse is not true in general. Consider $A=\QQ[X_0,X_1,\ldots]/(X_0^2,X_1^2,\ldots)$ and the polynomial $f(x):=\sum \ovl{X_i}x^i \in A[x]$.\footnote{David E. Fields. \textit{Zero Divisors and Nilpotent Elements in Power Series Rings}. Proceedings of the American Mathematical Society, 1971, Vol.27 (3), 427-433.} Clearly all the coefficients of $f(x)$ are nilpotent. We claim that $f(x)$ is not nilpotent.

For each $k\in\NN$, let us look at the degree $n:=k(k-1)/2$ term in the expansion of $f(x)^k$, which is
$$
\sum_{\substack{ e_0+\cdots+e_n = k \\ e_1+2e_2+\cdots+ne_n = n \\ e_i\geq0 }} c_{e_0,\ldots,e_n} \cdot \ovl{X_0}^{e_0} \ovl{X_1}^{e_1} \cdots \ovl{X_n}^{e_n} x^n
$$
where $e_i$ has its obvious meaning, and $c_{e_0,\ldots,e_n}$ is a natural number. But we see that if there exists some $e_i\geq 2$, then the corresponding coefficient will be zero in $A$. So we may only consider when $e_i=0$ or $1$ for all $i$, which is
$$
\sum_{\substack{ e_0+\cdots+e_n = k \\ e_1+2e_2+\cdots+ne_n = n \\ e_i=0,1 }} c_{e_0,\ldots,e_n} \cdot \ovl{X_0}^{e_0} \ovl{X_1}^{e_1} \cdots \ovl{X_n}^{e_n} x^n
$$

Note that the conditions on $e_0,\ldots,e_n$ now mean we're choosing $k$ numbers from the set $\{0,\ldots,n\}$ s.t. they sum up to $n$. It's not hard to see that the only possibility is when $e_i=1$ for $0 \leq i \leq k-1$ and $0$ otherwise. This implies that the coefficient of $x^n$ in the expansion of $f(x)^k$ is $c\ovl{X_0}\cdots\ovl{X_{k-1}}$ for some $c\in\NN$. And since $X_0\cdots X_{k-1} \not\in (X_0^2,X_1^2,\ldots)$, we have $\ovl{X_0}\cdots\ovl{X_{k-1}} \neq \ovl{0}$ in $A$ and hence $f(x)^k\neq0$.

(iii) By (1.9), it's equivalent to show that $1-fg$ is a unit in $A[[x]]$ for all $g\in A[[x]]$ if and only if $1-a_0a$ is a unit in $A$ for all $a\in A$.

$(\Rightarrow)$ Given $a\in A$, then by the assumption $1-f\cdot a = 1 - a_0a - \sum_{n\geq1} a_nax^n$ is a unit in $A[[x]]$. And by (i), this implies $1-a_0a$ is a unit in $A$.

$(\Leftarrow)$ Given $g=\sum b_mx^m \in A[[x]]$, then by the assumption $1-a_0b_0$ is a unit in $A$. And by (i) again, this implies $1-fg = 1-a_0b_0 - (\text{higher degree terms})$ is a unit in $A[[x]]$.

(iv) Consider the natural surjective ring homomorphisms
$$
\phi: A[[x]] \twoheadrightarrow A[[x]]/(x) \simeq A \twoheadrightarrow A/\mfk^c
$$
We check that $\ker\phi = \mfk^{ce} + xA[[x]]$. Given $f=\sum a_nx^n \in ker\phi$. This means $a_0 \in \mfk^c \sbe \mfk^{ce}$ and so $f = a_0 + \sum_{n\geq1} a_nx^n \in \mfk^{ce} + xA[[x]]$. Conversely, given $\sum m_if_i + xg \in \mfk^{ce} + xA[[x]]$ where $m_i\in\mfk^c$ and $f_i,g \in A[[x]]$. Then clearly the constant term is in $\mfk^c$ and so its image under $\phi$ is zero.

From the above, we have the ring isomorphism
$$
A[[x]]/ (\mfk^{ce} + xA[[x]]) \simeq A/\mfk^c
$$
Next, we show that $\mfk = \mfk^{ce} + xA[[x]]$. This will follow immediately if we show that $x\in\mfk$. So suppose not, then $\ovl{x}\neq\ovl{0}$ in the field $A[[x]]/\mfk$. So there exists $f\in A[[x]]$ s.t. $\ovl{xf}=\ovl{1}$ in $A[[x]]/\mfk$, or equivalently, $1-xf\in \mfk$. This implies $1-xf$ is not a unit in $A[[x]]$ and so by (i), $1$ is not a unit in $A$, which is absurd.

In conclusion, we have that $\mfk$ is generated by $\mfk^c$ and $x$. And
$$
A[[x]]/\mfk = A[[x]]/ (\mfk^{ce} + xA[[x]]) \simeq A/\mfk^c
$$
Since the $\lhs$ is a field by the assumption, hence so is the $\rhs$. This completes the proof.

(v) Let $\pfk\sbe A$ be a prime ideal. Similar to (iv) (changing the role of $\mfk^c$ to $\pfk$), we consider the natural surjective ring homomorphisms
$$
A[[x]] \twoheadrightarrow A[[x]]/(x) \simeq A \twoheadrightarrow A/\pfk
$$
with kernel $\pfk^e  + xA[[x]]$. So we have the ring isomorphism
$$
A[[x]]/ (\pfk^e + xA[[x]]) \simeq A/\pfk
$$
Since the $\rhs$ is an integral domain, so is the $\lhs$. Hence $\pfk^e + xA[[x]]$ is a prime ideal in $A[[x]]$. Moreover, one shows easily that $(\pfk^e+xA[[x]])^c = \pfk$. This completes the proof.

\subsection*{Exercise 1.6}

Since the Jacobson radical $\Jcal(A)$ (resp. nilradical $\Ncal(A)$) of $A$ is the intersection of all maximal (resp. prime) ideals in $A$, and all maximal ideals are necessarily prime ideals, so clearly $\Ncal(A) \sbe \Jcal(A)$.

Conversely, given $x\in \Jcal(A)$. If $x\notin\Ncal(A)$, then the ideal $(x)\nsubseteq\Ncal(A)$. By the assumption, $(x)$ contains a non-zero idempotent, say $0\neq ax\in(x)$. This means $(ax)^2=ax$, or equivalently, $ax(1-ax)=0$. So $1-ax$ is a zero divisor, which is always a non-unit. Hence by (1.9), $x\notin\Jcal(A)$, a contradiction.

\subsection*{Exercise 1.7}

Let $\pfk$ be a prime ideal of $A$ and given $\ovl{x}\in A/\pfk$. By the assumption we have $\ovl{x}^n=\ovl{x}$ in $A/\pfk$ for some $n>1$. Since $A/\pfk$ is an integral domain, this implies $\ovl{x}^{n-1}=\ovl{1}$. So $\ovl{x}\in A/\pfk$ is a unit with the inverse element $\ovl{x}^{n-2}$ and hence $A/\pfk$ is a field.

\subsection*{Exercise 1.8}

Let $\Sigma:=\{A\setminus\pfk \mid \pfk \text{ is a prime ideal}\}$ be the collection of all complements of prime ideals in $A$, which is non-empty because $A$ has a maximal ideal by (1.3). For any chain $S=(A\setminus\pfk_i)_{i\in I}$ in $\Sigma$, we claim that $\bigcup_{i\in I} (A\setminus\pfk_i) = A\setminus \bigcap_{i\in I} \pfk_i$ is an upper bound of $S$ in $\Sigma$.

Note it's enough to show that $\bigcap_{i\in I} \pfk_i$ is a prime ideal in $A$. Given $x,y \notin \bigcap_{i\in I} \pfk_i$, then $x\notin \pfk_i$ and $y\notin\pfk_j$ for some $i,j\in I$. Since $S$ is a chain, WLOG, we assume $A\setminus\pfk_i \spe A\setminus\pfk_j$. This implies $x,y\notin \pfk_i$ and so $xy\notin\pfk_i$ as $\pfk_i$ is a prime ideal. Hence $xy\notin \bigcap_{i\in I} \pfk_i$. This shows that $\bigcap_{i\in I} \pfk_i$ is a prime ideal in $A$.

By Zorn's lemma, $\Sigma$ has a maximal element, say $A\setminus\pfk \in \Sigma$. We check that $\pfk$ is minimal in the set of prime ideals of $A$. If $\pfk' \sbe \pfk$ is another prime ideal, then $A\setminus\pfk' \spe A\setminus\pfk$ and so $A\setminus\pfk' = A\setminus\pfk$ by the maximality. This implies $\pfk'=\pfk$ and hence $\pfk$ is minimal.

\subsection*{Exercise 1.9}

$(\Rightarrow)$ Suppose $\afk=r(\afk)$. By (1.14), $r(\afk)$ is the intersection of the prime ideals of $A$ containing $\afk$. So we are done.

$(\Rightarrow)$ Suppose $\afk=\bigcap \pfk_i$ is an intersection of prime ideals. Then by (1.13) (vi),
$$
r(\afk) = r\left( \bigcap \pfk_i \right) \sbe \bigcap r(\pfk_i) = \bigcap \pfk_i = \afk
$$
Since we already have $r(\afk)\spe\afk$, so $r(\afk)=\afk$.

\subsection*{Exercise 1.10}

(i) $\Rightarrow$ (ii) and (i) $\Rightarrow$ (iii) If $A$ has exactly one prime ideal $\pfk$, then $A$ is a local ring with the unique maximal ideal $\pfk$, and its nilradical $\Ncal(A)=\pfk$. These two imply that $A$ is a disjoint union of units and nilpotent elements, and $A/\Ncal(A)$ is a field.

(ii) $\Rightarrow$ (iii) If every element in $A$ is either a unit or a nilpotent element, then every non-zero element in $A/\Ncal(A)$ is represented by a unit in $A$, which is clearly a unit in $A/\Ncal(A)$.

(iii) $\Rightarrow$ (i) Suppose $A/\Ncal(A)$ is a field, then $\Ncal(A)$ is itself a prime ideal. Let $\pfk$ be any prime ideal in $A$. We claim that $\pfk=\Ncal(A)$. Clearly $\Ncal(A) \sbe \pfk$. If there exists $x\in\pfk$ but $x\notin\Ncal(A)$, then for some $y\in A$ we have $1-xy\in\Ncal(A) \sbe \Jcal(A)$, the Jacobson radical of $A$. By (1.9), $1-(1-xy) = xy$ is a unit. Hence so is $x$. But this is a contradiction as $x\in\pfk$. So we have $\pfk=\Ncal(A)$.

\subsection*{Exercise 1.11}

(i) Since $x+1 = (x+1)^2 = x^2 + 2x + 1 = x + 2x + 1$. So $2x=0$.

(ii) Let $\pfk$ be a prime ideal in $A$. For each $x\notin\pfk$, $x(1-x)=0\in\pfk$. So $1-x\in\pfk$. This means that $\ovl{x}=\ovl{1}$ in $A/\pfk$ for all $x\notin\pfk$ and so the integral domain $A/\pfk = \{\ovl{0},\ovl{1}\}$ has only two elements, which is therefore a field. So $\pfk$ is maximal.

(iii) It's enough to consider ideals generated by two elements. But we see that $(x,y)=(x+y+xy)$ for all $x,y\in A$. (Note that $(x+y+xy)\cdot x = x^2+xy+x^2y = x$ by (i).)

\subsection*{Exercise 1.12}

Let $A$ be a local ring and $x\in A$ be an idempotent. Then $x$ is either a nilpotent or a unit. (See Exercise 1.10 if necessary.)

Case 1: $x$ is nilpotent, say $x^n=0$. Then $x=x^2=\cdots=x^n=0$.

Case 2: $x$ is a unit. Take $y\in A$ s.t. $1=xyx$. Then $x=x^2y=xy=1$.

\subsection*{Exercise 1.13}

If $\afk:=\langle f(x_f) \mid f\in\Sigma \rangle = A$, then $1=a_1f_1(x_{f_1})+\cdots+a_nf_n(x_{f_n})$ for some $a_i\in A$ and $f_i(x)\in\Sigma$. For each $i=1,\ldots,n$, we let $E_i$ be a field extension over $K$ in which $f_i(x)$ has a root $\alpha_i$ (for example, $E_i:=K[x]/f_i(x)$ works). And let $E$ be the compositum of $E_1,\ldots,E_n$. Then we have a field extension $E/K$, and may view the above as an equation over $E$. Plugging $x_{f_i} = \alpha_i$ for all $i$ and $x_f=0$ otherwise gives us $1=0$, a contradiction.

\subsection*{Exercise 1.14}

Let $\Sigma$ be the set of all ideals in $A$ in which every element is a zero divisor. $\Sigma$ is non-empty because it contains the zero ideal. Moreover, if $S=(\afk_i)_{i\in I}$ is a chain in $\Sigma$, then $\bigcup_{i\in I} \afk_i$ is an upper bound of $S$ in $\Sigma$. So by Zorn's lemma, $\Sigma$ has a maximal element.

We next show that any maximal element $\afk$ in $\Sigma$ is a prime ideal. Suppose on the contrary that there exist $x,y\in A$ with $xy\in\afk$ but both $x,y\notin\afk$. Then $\afk+(x),\afk+(y) \supset \afk$. By the maximality of $\afk$, these two ideals are not in $\Sigma$. So there exist $a+xt\in\afk+(x)$ and $b+ys\in\afk+(y)$ which are not zero divisors. But note that $(a+xt)(b+ys) \in \afk$ as $xy\in\afk$. So their product is a zero divisor, which is absurd.

\subsection*{Exercise 1.15}

Let $\pfk\in X$ be any prime ideal.

(i) This is equivalent to show that $E \sbe \pfk \iff \afk \sbe \pfk \iff r(\afk) \sbe \pfk$. The first one is almost trivial. And for the second one, use (1.13) (i) and (vi).

(ii) Trivial.

(iii) This is equivalent to show that $\bigcup_{i\in I} E_i \sbe \pfk \iff E_i \sbe \pfk$ for all $i\in I$, which is also trivial.

(iv) For the first one, we use (i) and (1.13) (iii) to see that
$$
V(\afk \cap \bfk) = V(r(\afk \cap \bfk)) =  V(r(\afk\bfk)) = V(\afk\bfk) 
$$
For the second one, it's equivalent to show that $\afk\bfk \sbe \pfk \iff \afk \sbe \pfk$ or $\bfk \sbe \pfk$, which is again, trivial.

\subsection*{Exercise 1.16}

\subsection*{Exercise 1.17}

For $\pfk\in X:=\Spec(A)$, we see from the definition of $X_f$ that $\pfk\in X_f$ if and only if $f\notin\pfk$.

Let $\Bcal:=\{ X_f \mid f\in A\}$. We check that $\Bcal$ forms a basis for the Zariski topology. Let $U\sbe X$ be an open set and $\pfk\in U$. Then $U=X \setminus V(\afk)$ for some ideal $\afk$ in $A$ by Exercise 1.15 (i). Since $\pfk\in U \implies \pfk\notin V(\afk) \implies \afk\nsubseteq\pfk$. So we may choose $f\in\afk$ but $f\notin\pfk$. The latter implies $\pfk\in X_f$. And the former implies $(f)\sbe\afk \implies V(\afk)\sbe V(f) \implies U\spe X_f$. Hence we have $\pfk \in X_f \sbe U$.

(i) This is equivalent to show that both $f,g \notin \pfk \iff fg \notin \pfk$, which is clear.

(ii) $X_f=\emptyset \iff V(f)=X \iff f\in\pfk$ for all $\pfk \iff f\in \Ncal(A)$, the nilradical of $A \iff f$ is nilpotent.

(iii) $X_f=X \iff V(f)=\emptyset \iff f\notin\pfk$ for all $\pfk \iff f$ is a unit. (See (1.5) if necessarily.)

(iv) We show that $V(f)=V(g)$ if and only if $r((f))=r((g))$. Suppose $V(f)=V(g)$. Then by (1.14) and Exercise 1.15 (i), we have
$$
r((f)) = \bigcap_{\pfk\in V((f))} \pfk = \bigcap_{\pfk\in V((g))} \pfk = r((g))
$$
Conversely, suppose $r((f))=r((g))$. Then by Exercise 1.15 (i) again,
$$
V(f) = V((f)) = V(r((f))) = V(r((g))) = V((g)) = V(g)
$$

(v) Suppose $X$ is covered by basis elements $\{X_{f_i}\}_{i\in I} \sbe \Bcal$. Let $\afk:=\langle f_i \mid i\in I \rangle$, then
$$
X = \bigcup_{i\in I} X_{f_i} = \bigcup_{i\in I} (X\setminus V(f_i)) = X\setminus  \bigcap_{i\in I} V(f_i) = X \setminus V\left( \bigcup_{i\in I} \{f_i\} \right) = X\setminus V(\afk)
$$
by Exercise 1.15 (i) and (iii). So $V(\afk)=\emptyset$, i.e., $\afk\nsubseteq\pfk$ for all $\pfk$. This implies $\afk=A$ by (1.4).

Consequently, we have $1=\sum_{i\in J} g_if_i$ where $J\sbe I$ is a finite set. We check that $X=\bigcup_{i\in J} X_{f_i}$. Let $\pfk\in X$. Then $f_i\notin \pfk$ for some $i\in J$. (Otherwise, we have $1\in \pfk$, which is absurd.) So $\pfk\in X_{f_i} \sbe \bigcup_{i\in J} X_{f_i}$.

(vi) Suppose $X_f$ is covered by basis elements $\{X_{f_i}\}_{i\in I} \sbe \Bcal$. Let $\afk:=\langle f_i \mid i\in I \rangle$, then similar to (v), we have $X_f = X\setminus V(f) \sbe \bigcup_{i\in I} X_{f_i} = X\setminus V(\afk)$ and so $V(f)\spe V(\afk)$. And by (1.13) (i), (1.14) and Exercise 1.15 (i), this implies
$$
(f) \sbe r((f)) = \bigcap_{\pfk\in V(f)} \pfk \sbe \bigcap_{\pfk\in V(\afk)} \pfk = r(\afk)
$$
So we have $f^n\in\afk$ for some $n>0$.

Write $f^n=\sum_{i\in J} g_if_i$ where $J\sbe I$ is a finite set. By (i) we have $X_{f^n} = X_f$, so we check that $X_{f^n} \sbe \bigcup_{i\in J} X_{f_i}$. Let $\pfk\in X_{f^n}$. Then $f^n\notin\pfk$ and so $f_i\notin \pfk$ for some $i\in J$. (Otherwise, $f^n\in \pfk$, which is absurd.) So $\pfk\in X_{f_i} \sbe \bigcup_{i\in J} X_{f_i}$.

(vii) $(\Rightarrow)$ Let $U\sbe X$ be an open subset which is quasi-compact. Since $\Bcal$ is a basis of $X$, $U=\bigcup_{i\in I} X_{f_i}$ for some index set $I$. And since $U$ is quasi-compact, there exists a finite set $J\sbe I$ s.t. $U\sbe \bigcup_{i\in J} X_{f_i}$. Since each $X_{f_i} \sbe U$ so we have $U = \bigcup_{i\in J} X_{f_i}$.

$(\Leftarrow)$ Suppose $U = \bigcup_{i=1}^n X_{f_i}$. Since from (vi) we know each $X_{f_i}$ is quasi-compact, hence so is $U$.

\subsection*{Exercise 1.18}

(i) If $\{x\}$ is closed, then $\{x\}=V(\afk)$ for some proper ideal $\afk$ in $A$. So $\pfk_x$ is the only prime ideal containing $\afk$. Since $\afk$ is necessarily contained in some maximal ideal by (1.4), so $\pfk_x$ must be maximal. Conversely, if $\pfk_x$ is maximal, then $V(\pfk_x)=\{x\}$. So $\{x\}$ is closed.

(ii)
$$
\ovl{\{x\}} = \bigcap_{x\in V(\afk)} V(\afk) = V\left( \bigcup_{x\in V(\afk)} \afk \right) = V\left( \bigcup_{\afk\sbe \pfk_x} \afk \right) = V(\pfk_x)
$$
where $\afk$ runs through all ideals in $A$ with the indicated conditions. (We've used Exercise 1.15 (iii) in the second equality.)

(iii) By (ii), $y\in\ovl{\{x\}} = V(\pfk_x) \iff \pfk_x \sbe \pfk_y$.

(iv) If $x,y$ are distinct points in $X$, then $\pfk_x\neq\pfk_y$. So either there exists $f\in\pfk_x$ but $f\notin\pfk_y$, or $f\in\pfk_y$ but $f\notin\pfk_x$. For the former case, we have $x\notin X_f$ and $y\in X_f$. So $X_f$ is an open neighborhood of $y$ which does not contain $x$. The latter case is similar.

\subsection*{Exercise 1.19}

$(\Rightarrow)$ Given $f,g\notin\Ncal(A)$. So they are not nilpotent. By Exercise 1.17 (ii), both $X_f$ and $X_g$ are non-empty. And since $\Spec(A)$ is irreducible, so $X_f \cap X_g \neq \emptyset$, say $x\in X_f \cap X_g$. This means both $f,g\notin\pfk_x$ and so $fg\notin\pfk_x$. Since $\Ncal(A)\sbe \pfk_x$, so $fg\notin\Ncal(A)$.

$(\Leftarrow)$ It's enough to check that any two non-empty basis elements $X_f,X_g$ intersect. By Exercise 1.17 (ii), we know $f,g$ are not nilpotent, so $f,g\notin\Ncal(A)$. Since $\Ncal(A)$ is a prime ideal, so $fg\notin\Ncal(A)$. Hence $X_f \cap X_g = X_{fg}\neq\emptyset$. (We've used Exercise 1.17 (i).)

\subsection*{Exercise 1.20}

\subsection*{Exercise 1.21}

\begin{comment}
(i) $\qfk\in\phi^{*^{-1}}(X_f) \iff \phi^*(\qfk) = \phi^{-1}(\qfk) \in X_f \iff f\notin \phi^{-1}(\qfk) \iff \phi(f)\notin\qfk \iff \qfk\in Y_{\phi(f)}$.

(ii) Note that $\qfk\in\phi^{*^{-1}}(V(\afk)) \iff \phi^*(\qfk) = \phi^{-1}(\qfk) =\qfk^c \in V(\afk) \iff \afk\sbe q^c$. So it's remaining to show that $\afk\sbe \qfk^c$ if and only if $\afk^e \sbe \qfk$. By (1.17) (i), if $\afk\sbe\qfk^c$, then $\afk^e\sbe\qfk^{ce} \sbe \qfk$. And conversely, if $\afk^e \sbe \qfk$, then $\afk \sbe \afk^{ec} \sbe \qfk^c$.

(iii) First, we claim that $\phi^*(V(\bfk)) \sbe V(\bfk^c)$. (Then $\ovl{\phi^*(V(\bfk))}  \sbe V(\bfk^c)$.) If $\pfk \in \phi^*(V(\bfk))$, then there exists $\qfk\spe\bfk$ s.t. $\phi^*(\qfk)=\pfk$. So $\bfk^c \sbe \qfk^c = \phi^{-1}(\qfk) = \phi^*(\qfk)=\pfk$ and hence $\pfk\in V(\bfk^c)$.

Conversely, let $V(\afk)$ be any closed set s.t. $\phi^*(V(\bfk)) \sbe V(\afk)$. We claim that $V(\bfk^c) \sbe V(\afk)$. (Then $\ovl{\phi^*(V(\bfk))}  \spe V(\bfk^c)$.) Given $\pfk \in V(\bfk^c)$. By Exercise 1.15 (i) and (1.18), we have $V(\bfk^c) = V(r(\bfk^c)) = V(r(\bfk)^c)$. So $\pfk\in V(r(\bfk)^c)$. By (1.14) and (1.13) (i),
$$
\pfk \spe r(\bfk)^c = \left( \bigcap_{\qfk \in V(\bfk)} \qfk \right)^c = \bigcap_{\qfk \in V(\bfk)} \qfk^c = \bigcap_{\qfk \in V(\bfk)} \phi^*(\qfk) = \bigcap_{\pfk \in \phi^*(V(\bfk))} \pfk \spe \bigcap_{\pfk \in V(\afk)} \pfk = r(\afk) \spe \afk 
$$
Thus, $\pfk\in V(\afk)$.
\end{comment}

\subsection*{Exercise 1.22}

\subsection*{Exercise 1.23}

\subsection*{Exercise 1.24}

\subsection*{Exercise 1.25}

\subsection*{Exercise 1.26}

\subsection*{Exercise 1.27}

\subsection*{Exercise 1.28}
\phantom{}

\end{document}