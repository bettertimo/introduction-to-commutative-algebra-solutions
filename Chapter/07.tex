\documentclass[../A&M.tex]{subfiles}

\begin{document}

\chapter{Noetherian Rings}

\subsection*{Exercise 7.1}

\subsection*{Exercise 7.2}

\subsection*{Exercise 7.3}

\subsection*{Exercise 7.4}

\subsection*{Exercise 7.5}

\subsection*{Exercise 7.6}

\subsection*{Exercise 7.7}

\subsection*{Exercise 7.8}

\subsection*{Exercise 7.9}

\subsection*{Exercise 7.10}

\subsection*{Exercise 7.11}

\subsection*{Exercise 7.12}

Given an ascending chain $\afk_1 \sbe \afk_2 \sbe \cdots$ of ideals of $A$. Since $B$ is flat over $A$, so $B\otimes_A \afk_1 \sbe B \otimes_A \afk_2 \sbe \cdots$ is an ascending chain of ideals of $B$. As $B$ is Noetherian, we have $B\otimes_A \afk_1 \sbe \cdots \sbe B \otimes_A \afk_n = B \otimes_A \afk_{n+1} = \cdots$ for some $n\in\NN$. And as $B$ is faithfully flat over $A$, we have $\afk_1 \sbe \cdots \sbe \afk_n = \afk_{n+1} = \cdots$. (See \nameref{Bonus 3.3} if necessary.) Hence, $A$ is Noetherian.

\subsection*{Exercise 7.13}

\subsection*{Exercise 7.14}

$f \in r(\afk) \implies f^n \in\afk$ for some $n\in\NN \implies 0 = f^n(x) = f(x)^n$ for all $x \in V \implies f(x)=0$ for all $x \in V \implies f \in I(V)$.

Conversely, suppose $f \notin r(\afk)$. By (1.14) there exists a prime $\pfk$ in $A$ s.t. $f\notin \pfk \spe \afk$. Let $B:= A/\pfk$ (an integral domain) and $C:= B_{\ovl{f}} = B[1/\ovl{f}]$ where $\ovl{f} \neq \ovl{0}$ is the image of $f$ in $B$. Let $\mfk$ be a maximal ideal in $C$. Then we have
$$
k \inj A = k[t_1,\ldots,t_n] \sur B = A/\pfk \inj C = B_{\ovl{f}} \sur C/\mfk.
$$
As $C$ is a finitely generated $B$-algebra and $B$ is a finitely generated $k$-algebra, we have $C$ is a finitely generated $k$-algebra. Moreover, as $k$ is algebraically closed, we have by (7.10) that $k \simeq C/\mfk$.

Let $x_i$ be the image of $t_i \in A$ in $C/\mfk$, and $x:=(x_1,\ldots,x_n) \in k^n$ be the corresponding point. Then for any $g\in\afk$, its image in $C/\mfk$ is zero. So $g(x) = 0$ and hence $x\in V$. Moreover, since $\ovl{f} \in B$ becomes a unit in $C$, so the image of $f$ in $C/\mfk$ is not zero. Hence $f(x)\neq0$. These two imply that $f \notin I(V)$.

\subsection*{Exercise 7.15}

\subsection*{Exercise 7.16}

\subsection*{Exercise 7.17}

\subsection*{Exercise 7.18}

\subsection*{Exercise 7.19}

\subsection*{Exercise 7.20}

\subsection*{Exercise 7.21}

\subsection*{Exercise 7.22}

\subsection*{Exercise 7.23}

\subsection*{Exercise 7.24}

\subsection*{Exercise 7.25}

\subsection*{Exercise 7.26}

\subsection*{Exercise 7.27}
\phantom{}

\end{document}